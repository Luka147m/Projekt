\documentclass[zavrsnirad]{fer}
% Dodaj opciju upload za generiranje konačne verzije koja se učitava na FERWeb
% Add the option upload to generate the final version which is uploaded to FERWeb

\usepackage{blindtext}
\usepackage{float}
\usepackage{adjustbox}
\usepackage{listings}
\usepackage{dirtree}

% Define custom colors for syntax highlighting
\definecolor{mygreen}{rgb}{0,0.6,0}
\definecolor{mygray}{rgb}{0.5,0.5,0.5}
\definecolor{mymauve}{rgb}{0.58,0,0.82}
\definecolor{lightgray}{rgb}{.9,.9,.9}
\definecolor{darkgray}{rgb}{.4,.4,.4}
\definecolor{purple}{rgb}{0.65, 0.12, 0.82}
\definecolor{codegreen}{rgb}{0,0.6,0}
\definecolor{codegray}{rgb}{0.5,0.5,0.5}
\definecolor{codepurple}{rgb}{0.58,0,0.82}
\definecolor{backcolour}{rgb}{0.95,0.95,0.92}

\lstdefinelanguage{JSX}{
	keywords={return, if, else, for, while, do, switch, import, export, default, MapContainer, ZoomControl, TileLayer, Markers, MapRouteLine, MapClickHandler},
	basicstyle=\footnotesize\ttfamily,
	breaklines=true,
	keywordstyle=\color{blue},
	commentstyle=\color{codegreen},
	stringstyle=\color{codepurple},
	identifierstyle=\color{black},
	numbers=left,
	numberstyle=\tiny\color{codegray},
	tabsize=2,
	showstringspaces=false,
	captionpos=b
}

\lstdefinelanguage{JavaScript}{
	keywords={typeof, new, true, false, catch, function, return, null, catch, switch, var, if, in, while, do, else, case, break, try, async, await},
	keywordstyle=\color{blue}\bfseries,
	ndkeywords={class, export, boolean, throw, implements, import, this},
	ndkeywordstyle=\color{darkgray}\bfseries,
	identifierstyle=\color{black},
	sensitive=false,
	comment=[l]{//},
	morecomment=[s]{/*}{*/},
	commentstyle=\color{purple}\ttfamily,
	stringstyle=\color{red}\ttfamily,
	morestring=[b]',
	morestring=[b]"
}

% Define the style for listings
\lstdefinestyle{mystyle}{
	backgroundcolor=\color{lightgray},   % choose the background color
	basicstyle=\linespread{1.2}\ttfamily\footnotesize, % size of fonts used for the code
	breakatwhitespace=false,         % sets if automatic breaks should only happen at whitespace
	breaklines=true,                 % sets automatic line breaking
	captionpos=b,                    % sets the caption-position to bottom
	commentstyle=\color{mygreen},    % comment style
	escapeinside={\%*}{*)},          % if you want to add LaTeX within your code
	keywordstyle=\color{blue},       % keyword style
	stringstyle=\color{mymauve},     % string literal style
	numbers=left,                    % where to put the line-numbers; possible values are (none, left, right)
	numberstyle=\tiny\color{mygray}, % the style that is used for the line-numbers
	stepnumber=1,                    % the step between two line-numbers. If it's 1, each line will be numbered
	numbersep=5pt,                   % how far the line-numbers are from the code
	showspaces=false,                % show spaces adding particular underscores; it overrides 'showstringspaces'
	showstringspaces=false,          % underline spaces within strings only
	showtabs=false,                  % show tabs within strings adding particular underscores
	tabsize=2,                       % sets default tabsize to 2 spaces
}
% Apply the style to the listings package
\lstset{style=mystyle}
%--- PODACI O RADU / THESIS INFORMATION ----------------------------------------

% Naslov na engleskom jeziku / Title in English
\title{Visualization of public transport traffic in real time using the format
	GTFS}

% Naslov na hrvatskom jeziku / Title in Croatian
\naslov{Vizualizacija prometa javnog prijevoza u stvarnom vremenu uporabom formata
	GTFS}

% Broj rada / Thesis number
\brojrada{1246}

% Autor / Author
\author{Luka Miličević}

% Mentor 
\mentor{izv. prof. dr. sc.\@ Ivana Bosnić}

% Datum rada na engleskom jeziku / Date in English
\date{June, 2024}

% Datum rada na hrvatskom jeziku / Date in Croatian
\datum{lipanj, 2024.}

%-------------------------------------------------------------------------------


\begin{document}

% Naslovnica se automatski generira / Titlepage is automatically generated
\maketitle


%--- ZADATAK / THESIS ASSIGNMENT -----------------------------------------------

% Zadatak se ubacuje iz vanjske datoteke / Thesis assignment is included from external file
% Upiši ime PDF datoteke preuzete s FERWeb-a / Enter the filename of the PDF downloaded from FERWeb
\zadatak{figures/hr_0036543289_73.pdf}


%--- ZAHVALE / ACKNOWLEDGMENT --------------------------------------------------

\begin{zahvale}
  % Ovdje upišite zahvale / Write in the acknowledgment
Na samom početku želim izraziti svoju duboku zahvalnost svojoj obitelji i prijateljima za neizmjernu podršku i razumijevanje tijekom mog obrazovanja.
Posebnu zahvalnost dugujem svojoj mentorici, izv. prof. dr. sc. Ivani Bosnić, čije me stručno vodstvo i savjeti pratilo kroz cijeli proces izrade ovog rada. Vaša podrška bila je ključna za moj akademski napredak i uspjeh.
\end{zahvale}


% Odovud započinje numeriranje stranica / Page numbering starts from here
\mainmatter


% Sadržaj se automatski generira / Table of contents is automatically generated
\tableofcontents


%--- UVOD / INTRODUCTION -------------------------------------------------------
\chapter{Uvod}
\label{pog:uvod}

U velikim urbanim gradovima poput Zagreba, javni gradski prijevoz predstavlja neizostavan dio svakodnevnog života. On je temeljni stup mobilnosti, pružajući vitalnu infrastrukturu za povezivanje građana i omogućujući lakše i efikasnije kretanje unutar grada. Velika većina građana svakodnevno koristi javni prijevoz te s obzirom na važnost vremena u užurbanom ritmu gradskog života, dobra organizacija i planiranje putovanja su ključni, jer svi bi htjeli 15 minuta koje potroše na čekanju prijevoza iskoristiti na drukčiji način. Upravo s tim ciljem, ideja završnog rada je stvoriti web-aplikaciju, koja pruža praćenje tramvaja u stvarnom vremenu, olakšavajući korisnicima efikasno planiranje svojih putovanja, osiguravajući im precizne informacije o dolasku tramvaja i informacijama o pojedinim gradskim linijama prijevoza. Kroz ovaj zadatak, bit će istraženi tehnički izazovi koji nastupaju tijekom razvoja takve web-aplikacije. Dublje će se istražiti ključni elementi koji oblikuju temelje web-aplikacije poput analize API-ja kojeg pruža Zagrebački električni tramvaj (ZET), proučavanje otvorenih standarda \textit{GTFS} i \textit{GTFS realtime}, te detaljniju analizu strukture web-aplikacije, od baze podataka \textit{PostgreSQL}, poslužiteljskog dijela ostvarenog s razvojnim okvirima \textit{node.js} i \textit{Express.js}, do klijentskog dijela implementiranog s \textit{React.js} i bibliotekama poput \textit{Leaflet.js}.




%\begin{figure}[htb]
%  \centering
%  \includegraphics[width=0.38\linewidth]{Figures/lenovo_yoga_tab3_pro_front.png} 
%  \caption{Moja prva slika}
%  \label{slk:prvaslika}
%\end{figure}

%Referenciramo se na sliku \ref{slk:prvaslika} u sredini rečenice, zatim prije zareza %\ref{slk:prvaslika}, te zatim na kraju rečenice \ref{slk:prvaslika}.
%Upravo smo testirali radi li naredba \verb|\ref| ispravno u slučaju kada nakon nje slijedi točka.


%-------------------------------------------------------------------------------
\chapter{Zagrebački električni tramvaj}
Zagrebački električni tramvaj, poznatiji kao ZET, je trgovačko društvo koje je u direktnom
vlasništvu Grada Zagreba, te ima ključnu ulogu u organizaciji javnog gradskog prijevoza u Zagrebu.
Posluje od 1910. godine i svakodnevno pruža prijevozne usluge za gotovo milijun ljudi pomoću
tramvaja, autobusa, uspinjače i specijalnih prijevoza \cite{ZET}.
Nedavno je ZET otvorio svoje podatke javnosti pod Otvorenom dozvolom Republike Hrvatske. Ovi podaci obuhvaćaju \textit{GTFS static} i \textit{GTFS realtime} podatke, koji su trenutno parcijalno implementirani te su dostupni na službenoj stranici ZET-a \cite{ZET-GTFS}.
Ti podaci će biti temelj aplikacije i kroz njih ćemo detaljnije proći u poglavlju \ref{sec:zet-gtfs}
\\\\
\section{Slične aplikacije - ZET info}
Jedna popularna aplikacija koja koristi iste ZET-ove podatke je ZET info \cite{ZET-info}. ZET info primarno koristi podatke u formatu \textit{GTFS static} za prikaz rasporeda tramvaja ZET-a. Ona nije službena ZET-ova aplikacija te je dostupna preko trgovina Apple App Store i Google play. Vrlo je zgodna za imati pri ruci za pregledavanje rasporeda vožnji, ali bi aplikacijom ovog završnog rada htjeli ostvariti i upotrebu stvarnih (\textit{realtime}) podataka za tramvaje.

\begin{figure}[H]
	\centering
	\includegraphics[width=0.9\linewidth]{Figures/zetinfo.png} 
	\caption{Prikaz sučelja aplikacije ZET info}
	\label{slk:zet-info}
\end{figure}

\newpage
\chapter{\mbox{Format General Transit Feed Specification}}
\label{sec:GTFS}

General Transit Feed Specification (GTFS) je otvoreni standardizirani format za rasporede
javnog prijevoza i pripadajuće geografske informacije. Široko se koristi diljem svijeta te ga
podržava više od 10 tisuća agencija javnog prijevoza u preko 100 zemalja. Neke od platformi koje
koriste GTFS su Google Maps, Apple Maps, Moovit, OpenStreetMap i druge \cite{GTFS}. GTFS je započeo 2005. godine kao ideja za sporedni projekt Googleovog zaposlenika Chrisa Harrelsona te je danas \textit{de facto} standard industrije. GTFS se sastoji od dva glavna dijela: \textit{GTFS schedule} (\textit{static}) i \textit{GTFS realtime}.

Činjenica da je GTFS otvoreni standard omogućuje da agencije za prijevoz učine informacije dostupnima pomoću bilo kojeg od mnogih alata koji već podržavaju GTFS. Korisnici tako mogu preuzeti i obrađivati te informacije proizvoljnom aplikacijom koja podržava GTFS ili ako žele razviju svoje programe za to. Otvoreni standardi dovode do stvaranja podataka koji se mogu lako dijeliti i potiču daljnji razvitak.

Jedan zanimljiv primjer male aplikacije koja koristi GTFS podatke da prikaže područja na karti do kojih možete doći javnim prijevozom u određenom vremenu je otvoreni projekt Mapnificent koji je javno dostupan na Githubu \cite{Mapnificent}. Na slici \ref{slk:mapnificent-zagreb} je prikazano dokle je moguće doći javnim prijevozom od zgrade FER-a unutar 12 minuta. 

\begin{figure}[htb]
  \centering
  \includegraphics[width=0.9\linewidth]{Figures/mapnificent-zagreb.jpg} 
  \caption{Mapnificent Zagreb}
  \label{slk:mapnificent-zagreb}
\end{figure}
\newpage
\section{Format GTFS static}

\textit{GTFS static} ili \textit{GTFS schedule} je kolekcija od barem 6 osnovnih, do 26 CSV (comma-separated
values) datoteka s ekstenzijom \textit{.txt} zapakiranih unutar jedne komprimirane \textit{.zip} datoteke. Te datoteke
sadrže informacije o rutama, rasporedima, stanicama, i ostalim raznim informacijama o prijevozu \cite{GTFS-schedule}. Sam format datoteka koji je CSV, pruža otvorenost i jednostavnost rukovanja jer mnogo programskih alata i jezika ima metode za učinkovitu obradu CSV-datoteka što čini implementaciju strojnog čitanja lakoćom. Važno za napomenuti da je način kodiranja datoteka imperativan, obavezuje se korištenje UTF-8 kodiranja (podržan je i UTF-8 BOM).


Na dijagramu \ref{slk:gtfs-model} prikazan je jedan primjer važeće strukture \textit{GTFS static} modela kojeg prikladno prikazuju datoteke u \textit{Entity–relationship} (ER) modelu. Zelenom bojom su označene neophodne datoteke, a plavom nekoliko dodatnih opcionalnih datoteka te za svaku od njih struktura i međusobne relacije. 

\begin{figure}[H]
	\centering
	\includegraphics[width=\linewidth-30pt]{Figures/gtfs-model.png}
	\caption{Model specifikacije GTFS static \cite{GTFS-schedule-model}}
	\label{slk:gtfs-model}
\end{figure}

\newpage

\subsection{Obavezne datoteke}
Ove datoteke su obavezne i moraju biti definirane da bismo zadovoljili GTFS standard. Spojene zajedno, pomoću pojedinih identifikatora, one čine cjelovitu sliku mreže javnog prijevoza:
\begin{itemize}
	\item \textbf{Agency.txt} -
	Opće informacije o prijevoznoj agenciji, uključujući njezino ime, web stranicu, i vremensku zonu.
	\item \textbf{Routes.txt} -
	Definira različite linije javnog prijevoza. Svaka ruta obuhvaća informacije o identifikatorima ruta, tipovima vozila i drugim detaljima.
	\item \textbf{Trips.txt} -
	Informacije o specifičnim putovanjima na određenim rutama, uključujući vremena polaska i druge povezane podatke.
	\item \textbf{Stop\_times.txt} -
	Informacije o vremenima dolaska i odlaska vozila na svakoj stanici tijekom svakog putovanja.
	\item \textbf{Stops.txt} -
	Informacije o stanicama, uključujući identifikatore, nazive i geografske koordinate.
	\item \textbf{Calendar.txt} -
	Informacije o tjednom rasporedu putovanja.
\end{itemize}

\subsection{Opcionalne datoteke}
Pored neophodnih datoteka korisno je imati i dodatne opcionalne datoteke poput:
\begin{itemize}
	\item \textbf{Shapes.txt} -
	Informacije o geografskim koordinatama rute koje omogućuju lijepo iscrtavanje rute na karti.
	\item \textbf{Transfers.txt} -
	Informacije o povezivanju više ruta kako bi se olakšala mogućnost presjedanja i kombiniranja prijevoznih linija.
	\item \textbf{Calendar\_dates.txt} -
	Omogućuje definiranje iznimaka u redovnom rasporedu kada usluga radi ili ne radi na pojedinoj liniji zbog smetnji ili radova, ili pak posebnih prijevoza.
\end{itemize}

\section{Format GTFS realtime}

\textit{GTFS realtime} ili \textit{GTFS-rt} je ekstenzija GTFS-a koja je nastala 2011. godine te je također \textit{de facto}
industrijski standard za dijeljenje stvarnih podataka o prijevozu. S obzirom da je GTFS otvoreni standard,
\textit{realtime} se razvio s ciljem maksimalne interoperabilnosti i lakoće korištenja da bi programerima bio olakšan razvoj aplikacija i dijeljenja informacija o prijevozu između agencija. \textit{GTFS-rt}
sadrži informacije u stvarnom vremenu o lokacijama vozila, predviđenim vremenima dolaska te
obavijestima o promjenama ruta i otkazivanjima putem poslužitelja weba (putem nekog API-ja) koji koristi \textit{protocol buffers} (\textit{protobuf}, poglavlje \ref{sec:protobuf}). Agencije sustavom za automatsko praćenje vozila neprekidno stvaraju podatke o lokacijama vozila dok se vremena dolaska na odredište najčešće izračunavaju pomoću modela strojnog učenja koji analiziraju povijesne podatke o položaju i voznom redu te daje očekivana vremena. Upravo jer se podaci neprestano kreiraju koriste se \textit{protocol buffers} koji jako brzo i efektivno rade binarnu serijalizaciju podataka u male pakete koji se lako šalju \cite{GTFS-realtime}.

Glavni "objekt" \textit{GTFS realtime} feed-a (stream) je \textit{FeedMessage} koji sadrži \textit{FeedHeader} i \textit{FeedEntity}.
\textit{FeedHeader} sadrži osnovne informacije o podacima (metapodatke) poput verzije GTFS-a, "\textit{incrementiality}" koji opisuje šalje li se cijeli skup podataka (\textit{dataset}) ili samo razlika od posljednje verzije te vremenska oznaka (\textit{timestamp}) kada su podaci generirani.
\textit{FeedEntity} sadrži jedan ili više entiteta koji sadrže najnovije informacije o stanju puta, pozicijama vozila, promjenama na rutama i upozorenjima. Neke od vrsta objekta \textit{FeedEntity} su: 

\begin{itemize}
	\item \textbf{tripUpdate} - predstavlja informacije jednog putovanja te sadrži objekte \textit{trip} (osnovne informacije o ruti/putu), \textit{stopTimeUpdate}, \textit{vehicle} (informacije o položaju vozila npr. geografske koordinate) i druge.
	\item \textbf{stopTimeUpdate} - nalazi se unutar \textit{tripUpdate} i on sadrži informacije o dolascima/odlascima sa stanica te kašnjenja.
	\item \textbf{alert} - sadrži upozorenja od nesreća, radova na cesti i njihovih obilazaka, vremenskih upozorenja i drugih raznih iznimaka.
\end{itemize}

Primjer GTFS-rt podataka se može vidjeti u sljedećem poglavlju na slikama \ref{slk:reply1} i \ref{slk:reply2}.

\section{Podaci ZET-a u formatu GTFS}
\label{sec:zet-gtfs}

Kao što je već rečeno ZET-ovi podaci su dostupni na službenoj stranici svima pod Otvorenom dozvolom Republike Hrvatske \cite{ZET-GTFS}. Na slici \ref{slk:zet-stranica} je prikazana ta stranica te na njoj se mogu pronaći poveznice za \textit{GTFS static} i \textit{GTFS realtime} podatke.

U narednom tekstu bit će objašnjeni ZET-ovi \textit{GTFS static} podaci, a potom i \textit{GTFS realtime} podatci. Kada se podaci dohvate sa stranice, dobivenu arhivu treba otvoriti s odgovarajućim alatom za pregled komprimiranih arhiva. Kada se to napravi dobivaju se datoteke sa slike \ref{slk:zet-podaci}.

\begin{figure}[htb]
	\centering
	\includegraphics[width=0.7\linewidth]{Figures/zet-stranica.jpg} 
	\caption{ZET stranica za GTFS podatke}
	\label{slk:zet-stranica}
\end{figure} 

\begin{figure}[htb]
	\centering
	\includegraphics[width=0.7\linewidth]{Figures/zet-podaci.jpg} 
	\caption{ZET \textit{GTFS static} podaci}
	\label{slk:zet-podaci}
\end{figure} 

Kao što vidimo tu su sve neophodne datoteke (\textit{trips.txt}, \textit{stops.txt}, \textit{stop\_times.txt}, \newline\textit{routes.txt}, \textit{calendar.txt} i \textit{agency.txt}) te nekoliko dodatnih datoteka \textit{feed\_info.txt} i \newline\textit{calendar\_dates.txt} koje nisu pretežito korisne za implementaciju aplikacije. Opcionalna datoteka \textit{feed\_info.txt} sadrži opće informacije o GTFS \textit{feed}-u, odnosno  informacije poput naziva agencije koja je stvorila \textit{feed}, vremena kada je \textit{feed} generira, verzije GTFS-standarda koja se koristi i slično.

Potreban je pobliži pregled datoteka \textit{trips.txt}, \textit{stops.txt}, \textit{stop\_times.txt} i \textit{routes.txt} jer one sadrže temeljne podatke za rad aplikacije.

\subsubsection{trips.txt}

% Please add the following required packages to your document preamble:
% \usepackage[table,xcdraw]{xcolor}
% Beamer presentation requires \usepackage{colortbl} instead of \usepackage[table,xcdraw]{xcolor}
\begin{table}[htb]
	\begin{adjustbox}{width=\columnwidth,center}
	\begin{tabular}{l|l|l|l|l|l|l|l}
		\hline
		\multicolumn{1}{c|}{\textbf{route\_id}} & \multicolumn{1}{c|}{\textbf{service\_id}} & \multicolumn{1}{c|}{\textbf{trip\_id}} & \multicolumn{1}{c|}{\textbf{trip\_headsign}} & \multicolumn{1}{c|}{\textbf{trip\_short\_name}} & \multicolumn{1}{c|}{\textbf{direction\_id}} & \multicolumn{1}{c|}{\textbf{block\_id}} & \multicolumn{1}{c}{\textbf{shape\_id}} \\ \hline
		268 & "0\_21" & "0\_21\_26820\_268\_10001" & "V. Gorica" &  & \textbf{0} & 26820 &  \\ \hline
		8 & "0\_21" & "0\_21\_801\_8\_10002" & "Grač. dolje" & \textbf{} & 1 & 801 &  \\ \hline
		7 & "0\_21" & "0\_21\_701\_7\_10012" & "Arena Zagreb" & \textbf{} & 1 & 701 &  \\ \hline
	\end{tabular}
	\end{adjustbox}
	\caption{Primjer datoteke trips.txt}
	\label{tbl:trips}
\end{table}


Bitne informacije u \textit{trips.txt} su:
\begin{itemize}
	\item \textbf{route\_id} - identifikator rute, odnosno broj linije (npr. Kod 17 Borongaj, to je 17)
	\item \textbf{trip\_id} - identifikator putovanja na pojedinoj ruti (svaka ruta ima više tramvaja koji više puta dnevno putuju na toj ruti)
	\item \textbf{trip\_short\_name} i \textbf{direction\_id} - ime smjera tramvaja i binarni broj koji određuje smjer (npr. Borongaj)
\end{itemize}

Također možemo primijetiti da u datoteci nedostaje \textit{shape\_id} koji se koristi kao ključ za relaciju \textit{trips.txt} i \textit{shapes.txt} koja kao što je rečeno sadrži detaljne koordinate putanje pojedine rute, s kojom bi se na karti s lakoćom mogao iscrtati put glatkom neisprekidanom krivuljom.
Ali u ZET-ovim podacima se ne može pronaći niti \textit{shape\_id} niti \textit{shapes.txt} te su, kao približno rješenje tog problema, uzete koordinate svih stanica na pojedinoj ruti koje su spojene ravnim linijom. Izgled te linije nije idealan, ali je rješenje približno. Postoje i određene \textit{routing} komponente, ali i takvo rješenje ima svoje nedostatke, ali o tom više u Poglavlju \ref{sec:frontend} o implementaciji klijentske strane aplikacije.
\newpage
\subsubsection{stops.txt}

\begin{table}[htb]
	\begin{adjustbox}{width=\columnwidth,center}
	\begin{tabular}{l|l|l|l|l|l|l|l|l|l}
		\hline
		\multicolumn{1}{c|}{\textbf{stop\_id}} & \multicolumn{1}{c|}{\textbf{stop\_code}} & \multicolumn{1}{c|}{\textbf{stop\_name}} & \multicolumn{1}{c|}{\textbf{stop\_desc}} & \multicolumn{1}{c|}{\textbf{stop\_lat}} & \multicolumn{1}{c|}{\textbf{stop\_lon}} & \multicolumn{1}{c|}{\textbf{zone\_id}} & \multicolumn{1}{c|}{\textbf{stop\_url}} & \textbf{location\_type} & \textbf{parent\_station} \\ \hline
		"98" &  & "Črnomerec" &  & 45.815175 & 15.934464 &  &  & 1 &  \\ \hline
		"100" &  & "Sveti Duh" &  & 45.813468 & 15.942297 &  &  & 1 &  \\ \hline
		"106\_1" & "1" & "Trg J. Jelačića" &  & 45.812906 & 15.977312 &  &  & 0 & 106 \\ \hline
		"106" &  & "Trg J. Jelačića" &  & 45.813038 & 15.976928 &  &  & 1 &  \\ \hline
	\end{tabular}
	\end{adjustbox}
	\caption{Izgled stops.txt}
	\label{tbl:stops}
\end{table}

Bitne informacije u \textit{stops.txt} su:
\begin{itemize}
	\item \textbf{stop\_id} - identifikator pojedine stanice
	\item \textbf{stop\_name} - naziv stanice
	\item \textbf{stop\_lat} i \textbf{stop\_lon} - koordinate stanice
\end{itemize}


\subsubsection{stop\_times.txt}

\begin{table}[htb]
	\begin{adjustbox}{width=\columnwidth,center}
		\begin{tabular}{l|l|l|l|l|l|l|l|l}
			\hline
			\multicolumn{1}{c|}{\textbf{trip\_id}} & \multicolumn{1}{c|}{\textbf{arrival\_time}} & \multicolumn{1}{c|}{\textbf{departure\_time}} & \multicolumn{1}{c|}{\textbf{stop\_id}} & \multicolumn{1}{c|}{\textbf{stop\_sequence}} & \multicolumn{1}{c|}{\textbf{stop\_headsign}} & \multicolumn{1}{c|}{\textbf{pickup\_type}} & \multicolumn{1}{c|}{\textbf{drop\_off\_type}} & \textbf{shape\_dist\_traveled} \\ \hline
			"0\_21\_26820\_268\_10001" & "00:30:00" & "00:30:00" & "110\_82" & 1 & "V. Gorica" &  &  &  \\ \hline
			"0\_21\_26820\_268\_10001" & "00:32:30" & "00:32:30" & "238\_24" & 2 & "V. Gorica" &  &  &  \\ \hline
			"0\_21\_26820\_268\_10001" & "00:33:40" & "00:33:40" & "450\_24" & 3 & "V. Gorica" &  &  &  \\ \hline
			"0\_21\_26820\_268\_10001" & "00:35:20" & "00:35:20" & "1085\_24" & 4 & "V. Gorica" &  &  &  \\ \hline
		\end{tabular}
	\end{adjustbox}
	\caption{Izgled stop\_times.txt}
	\label{tbl:stop_times}
\end{table}

Bitne informacije u \textit{stop\_times.txt} su:
\begin{itemize}
	\item \textbf{trip\_id} - identifikator putovanja na pojedinoj ruti
	\item \textbf{arrival\_time} i \textbf{departure\_time} - vrijeme dolaska i odlaska
	\item \textbf{stop\_id} - identifikator pojedine stanice
	\item \textbf{stop\_sequence} - redni broj stanice na putu
\end{itemize}
Ovo je najveća datoteka s oko 70 MB (\textasciitilde{}1000000 linija) koja za svako pojedino putovanje (\textit{trip\_id}) sadrži informacije o svim stanicama na putu, očekivanom vremenu dolaska i odlaska sa stanice i redoslijedu stanica. Ona će biti ključna za iscrtavanje rute i predviđanje lokacije tramvaja.
\newpage
\subsubsection{routes.txt}

\begin{table}[htb]
	\begin{adjustbox}{width=\columnwidth,center}
	\begin{tabular}{l|l|l|l|l|l|l|l|l}
		\hline
		\multicolumn{1}{c|}{\textbf{route\_id}} & \multicolumn{1}{c|}{\textbf{agency\_id}} & \multicolumn{1}{c|}{\textbf{route\_short\_name}} & \multicolumn{1}{c|}{\textbf{route\_long\_name}} & \multicolumn{1}{c|}{\textbf{route\_desc}} & \multicolumn{1}{c|}{\textbf{route\_type}} & \multicolumn{1}{c|}{\textbf{route\_url}} & \multicolumn{1}{c|}{\textbf{route\_color}} & \textbf{route\_text\_color} \\ \hline
		1 & 0 & "1" & "Zap.kol. - Borongaj" &  & 0 &  & "ffffff" & "000000" \\ \hline
		2 & 0 & "2" & "Črnomerec - Savišće" &  & 0 &  & "ffffff" & "000000" \\ \hline
		3 & 0 & "3" & "Ljubljanica -Savišće" &  & 0 &  & "ffffff" & "000000" \\ \hline
		4 & 0 & "4" & "Savski most - Dubec" &  & 0 &  & "ffffff" & "000000" \\ \hline
	\end{tabular}
	\end{adjustbox}
	\caption{Izgled routes.txt}
	\label{tbl:routes}
\end{table}

Ovdje nam je jedina nova bitna informacija \textbf{route\_long\_name} koja nam daje puno ime rute (početno mjesto i krajnje mjesto).\\

To bi bile najbitnije \textit{GTFS static} informacije koje su upotrijebljene za izradu aplikacije, koje su od posebne značajnosti za izradu baze podataka koja sprema te odvojene datoteke kao tablice i spaja ih s odgovarajućim ključevima i time omogućuje lako i brzo pretraživanje te formuliranje odgovora.

\newpage
\section{GTFS realtime podaci Zet-a}
Na ZET-ovoj stranici nalazi se link za \textit{GTFS realtime} podatke \ref{slk:zet-stranica} koji kada se skinu su binarna datoteka u \textit{.proto} formatu. \textit{Protobuf} je opisan u narednom poglavlju \ref{sec:protobuf}. Ovdje su opisani samo konačni podaci koji se dobivaju deserijalizacijom.\\\\
Na slici \ref{slk:reply1} se nalazi jedan mali isječak od inače jako velike poruke (obično sadrže oko 500 - 600 entiteta), ali radi jednostavnosti je prikazan samo jedan.\\\\
Na početku svake poruke nalazi se \textit{header} te možemo primijetiti da se koristi starija verzija GTFS 1.0. Sada je aktualna verzija 2.0 koja ima bolje definiranu \textit{schemu} s više informacija i pruža veću fleksibilnost. Također piše da se svaki put šalju svi podaci "FULL\_DATASET" i na kraju je vremenska oznaka generiranja poruke koja je u UNIX-formatu.\\\\
Slijedi polje \textit{entity} koje sadrži entitete (objekte) koji reprezentiraju informacije o jednom vozilu. Svaki od njih ima jedinstveni \textit{ID} i \textit{tripUpdate}. \textit{TripUpdate} se sastoji od osnovnih informacija o putu na kojemu je vozilo, \textit{ID} tog puta, kada je započeo put, je li put prema rasporedu, broj rute kojoj pripada taj put, vremenska oznaka generiranja tih podataka i \textit{stopTimeUpdate} koji sadrži informacije o zadnjoj posjećenoj stanici njen ID, redni broj i vremenska oznaka.\\\\
Jedna stvar koju je vrijedno napomenuti da su ZET-ovi podaci prilično osnovni to jest sadrže najosnovnije informacije. Vjerojatni razlog toga je što su ZET-ovi podaci u fazi razvoja i na stranici je navedeno da je ovo prototip. Na primjer jedna jako velika stvar koja bi bila jako korisna jesu \textit{vehicle} podaci (slika \ref{slk:reply2}) koji sadrže točne geografske lokacije vozila. Trenutačno je aplikacija napravljena tako da na karti prikaže zadnje potvrđene pozicije tramvaja koje su zapravo podaci iz \textit{stopTimeUpdate} odnosno lokacije zadnje posjećenih stanica i markeri koji su predikcije položaja tramvaja temeljene na vremenima iz \textit{GTFS static} podataka. Implementacija tih \textit{vehicle} podataka o vozilima bi izbacilo potrebu za predikcijom i puno olakšala direktno prikazivanje točnih lokacija tramvaja.

\begin{figure}[H]
	\centering
	

	\begin{minipage}[htb]{0.58\linewidth}
		\centering
		
		\begin{lstlisting}[language=Java]
"header": {
	"gtfsRealtimeVersion": "1.0",
	"incrementality": "FULL_DATASET",
	"timestamp": "1702666249"
},
"entity": [
{
	"id": "XNYGO5S0BU",
	"tripUpdate": {
		"trip": {
			"tripId": "0_1_11504_115_10447",
			"startDate": "20231215",
			"scheduleRelationship": "SCHEDULED",
			"routeId": "115"
		},
		"stopTimeUpdate": [
		{
			"stopSequence": 12,
			"arrival": {
				"delay": 145,
				"time": "1702666205"
			},
			"stopId": "1634_22",
			"scheduleRelationship": "SCHEDULED"
		}
		],
		"timestamp": "1702666189"
	}
}
]
		\end{lstlisting} 
		\caption{Dio ZET-ovog GTFS-rt feeda}
		\label{slk:reply1}
	\end{minipage}
	\hfill
	\begin{minipage}[htb]{0.38\linewidth}
		\centering
		\begin{lstlisting}[language=Java]
			"vehicle": {
				"position": {
					"latitude": 45.8150,
					"longitude": 15.9819,
					"bearing": 120,
					"odometer": 15000.5,
					"speed": 30.5
				}
			}
		\end{lstlisting}
		\caption{Primjer vehicle podataka}
		\label{slk:reply2}
	\end{minipage}
\end{figure}


\newpage
\chapter{Mehanizam Protocol buffers}
\label{sec:protobuf}

Protocol Buffers su jezično i platformski neutralni, proširivi mehanizmi za serijalizaciju strukturiranih podataka. Google ih je razvio prvotno za internu uporabu, ali su kasnije postali dostupni pod otvorenom licencom \cite{protobuf}. Glavni cilj protocol buffera je pružiti jednostavnost i visoke performanse, te su posebno dizajnirani kako bi bili manji i brži od formata XML i JSON.


Protobuf koristi \textit{.proto} datoteke za definiranje strukture podataka koji koriste posebnu sintaksu. Trenutno su podržane verzije \textit{proto2} i \textit{proto3} u kojima se s ključnom riječi \textit{message} definira struktura podataka poruke. Na slici \ref{slk:proto} je mali primjer jedne definicije poruke \textit{Person} koja sadrži tri polja za podatke \textit{id, name i email}. Svaka od njih mora imati tip podatka i vrijednost koja služi za identifikaciju polja kod binarne serijalizacije. Neki tipovi podataka koji su podržani su \textit{int32, int64, float, double, bool, string, enum} i ugrađene poruke.

\begin{figure}[H]
	\centering
	\begin{minipage}{0.6\linewidth}
		\begin{lstlisting}[language=Python]
			syntax = "proto3";
			message Person{
				int32 id = 1;
				string name = 2;
				string email= 3;
			}
		\end{lstlisting}
	\end{minipage}
	\caption{Primjer definicije datoteke .proto}
	\label{slk:proto}
\end{figure}
\newpage		
	Nakon definiranja strukture poruke potrebno je iskoristiti proto compiler (\textit{protoc}) da bi generirali k\^od koji se koristi za serijalizaciju i deserijalizaciju podataka u odabranom programskom jeziku (podržani su C++, Java, Python, Go, JavaScript i mnogo drugih). Za primjer sa slike \ref{slk:proto} i za programski jezik Python, \textit{protoc} compiler generira modul \textit{person\_pb2} koji sadrži potrebne metode za kreiranje objekata te strukture, serijalizaciju i deserijalizaciju tih podataka. 

\begin{figure}[H]
	\centering
	\begin{minipage}{0.8\linewidth}
		\begin{lstlisting}[language=Python]
			import person_pb2
			
			person = person_pb2.Person()
			person.id = 123
			person.name = "John Doe"
			person.email = "johndoe@example.com"
			
			# Serijalizacija
			serialized_data = person.SerializeToString()
			
			# Deserijalizacija
			new_person = person_pb2.Person()
			new_person.ParseFromString(serialized_data)
		\end{lstlisting}
	\end{minipage}
	\caption{Python k\^od s generiranim proto modulom}
	\label{slk:proto_kod}
\end{figure}

Neke prednosti korištenja mehanizama \textit{protocol buffer} za razliku od tradicionalnijih opcija poput XML-a i JSON-a su:

\begin{itemize}
	\item Protobuf koristi binarni format za serijalizaciju, koji je mnogo kompaktniji i brži za parsiranje.
	\item Dizajniran s \textit{backward} i \textit{forwards compatibility} na umu 
	\item Međujezična kompatibilnost (\textit{Cross-Language compatibility})
	\item Imaju validaciju schema koji striktno čuvaju strukturu podataka
\end{itemize}

\newpage
\chapter{Arhitektura sustava}

Sustav se temelji na modelu klijent-poslužitelj. Taj model je danas jedan od najčešće korištenih arhitektura za raspodijeljene sustave. On se sastoji od dva dijela, poslužitelja koji nudi uslugu i klijenta koji traži uslugu. Klijent će biti korisnikovo računalo koje pomoću preglednika weba koristi klijentski dio aplikacije (\textit{React.js} frontend) te putem protokola HTTP komunikacijskog kanala komunicira s poslužiteljskim dijelom aplikacije (\textit{node.js} i \textit{PostgreSQL}).\\\\
Klijentski dio aplikacije je izrađen pomoću razvojnog okvira \textit{React.js}, popularnog JavaScript okvira za izgradnju korisničkih sučelja, primarno korišten za jednostranične aplikacije (Single-page Application, SPA).\\\\
Poslužiteljski dio aplikacije izrađen je pomoću \textit{node.js}, koji je također JavaScript okvir, ali za izgradnju aplikacija na strani poslužitelja. Aplikacija će također komunicirati i s relacijskom bazom podataka \textit{PostgreSQL}. Poslužitelj će obrađivati klijentske zahtjeve i pružati odgovarajuće odgovore nazad.\\

\begin{figure}[htb]
	\centering
	\includegraphics[width=0.8\linewidth]{Figures/arhitektura.png} 
	\caption{Arhitektura sustava}
	\label{slk:sustav}
\end{figure}
\newpage
\section{Baza podataka}
\label{sec:baza}
Baza podataka je izvedena pomoću sustava \textit{PostgreSQL}-a. \textit{PostgreSQL} je besplatan i otvoreni, objektno-relacijski sustav za upravljanje bazama podataka. Temelji se na SQL jeziku i nudi brojna proširenja koji olakšavaju pohranu i skaliranje, povećavaju sigurnost te rješavaju probleme paralelnosti pristupa. Nastao je 1986. godine na Kalifornijskom sveučilištu u Berkeleyju. \cite{postgresql}\\

Baza podataka je korištena zbog velike količine podataka iz ZET-ovih statičnih podataka te bi optimalan način upravljanja i pohranom tih podataka bila baza podataka.
Glavne uloge baze podataka u aplikaciji su:
\begin{itemize}
	\item Spremanje ZET-ovih statičnih podataka kao tablice
 	\item Povezivanje tablica statičnih podataka pomoću ključeva u virtualne tablice (\textit{views})
 	\item Funkcije za brzi pronalazak traženih podataka prema nekom parametru
 	\item Formiranje odgovora
\end{itemize}

Ubacivanje podataka iz ZET-ovih datoteka u bazu podataka je automatizirano pomoću python skripte (opisana u narednom poglavlju \ref{sec:skripta}). Datoteke unesene u bazu su \textit{trips.txt, stops.txt, routes.txt i stop\_times.txt}. One su ubačene kao najobičnije tablice gdje stupce pojedine tablice definira \textit{header} te datoteke (prisjetimo se, one su zapravo CSV datoteke! \ref{sec:zet-gtfs}).

Spajanje tih tablica je ostvareno virtualnim tablicama tzv. pogledi (\textit{views}) koji objedinjuju nekoliko tablica u jednu i ostavljaju samo navedene stupce. Sadržaj virtualne tablice dinamički se određuje u trenutku obavljanja operacije nad virtualnom tablicom te se ponovo izvode svaki put kada ih se pozove. Time se izbjegava problem "zastarijevanja" podataka.
\newpage
Neki od pogleda su \textit{stopsjson, tripstops i triptime}.
\textit{Stopsjson} vraća polje \textit{key-value} vrijednosti, gdje je ključ \textit{stop\_id} i vrijednost objekt s imenom i pozicijom stanice. Rezultat vraća kao JSON objekt.
\begin{figure}[h]
	\centering
	\begin{minipage}{0.8\linewidth}
		\begin{lstlisting}[language=SQL]
 		SELECT jsonb_object_agg(stop_id::text, jsonb_build_object('stop_name', stop_name, 'stop_lat', stop_lat, 'stop_lon', stop_lon)) AS stops
		FROM stops;
		\end{lstlisting}
	\end{minipage}
	\caption{K\^od pogleda \textit{stopsjson}}
	\label{slk:stopsjson}
\end{figure}
\\

Pogled \textit{tripstops} spaja tablice \textit{stop\_times} i \textit{stops} pomoću \textit{stop\_id} kao ključa. Objedinjuje informacije tih dviju tablica kako bismo imali potpune informacije o putu poput redoslijeda stanica, njihovih naziva i lokacija te vremena dolaska i odlaska.
\begin{figure}[h]
	\centering
	\begin{minipage}{0.8\linewidth}
		\begin{lstlisting}[language=SQL]
 		SELECT  stop_times.trip_id,
				stops.stop_id,
				stop_times.stop_sequence,
				stops.stop_name,
				stops.stop_lat,
				stops.stop_lon,
				stop_times.arrival_time,
				stop_times.departure_time,
				stop_times.stop_headsign
				FROM stop_times
				JOIN stops USING (stop_id)
				ORDER BY stop_times.stop_sequence;
		\end{lstlisting}
	\end{minipage}
	\caption{K\^od pogleda \textit{tripstops}}
	\label{slk:tripstops}
\end{figure}

Pogled \textit{triptime} filtrira tablicu \textit{tripstops} kako bi izračunali vrijeme potrebno do iduće stanice.
\begin{figure}[h]
	\centering
	\begin{minipage}{0.8\linewidth}
		\begin{lstlisting}[language=SQL]
			SELECT trip_id, stop_sequence,
			COALESCE(lead(arrival_time) OVER () - arrival_time, '00:00:00'::interval) AS time_till_next_stop
			FROM tripstops
			ORDER BY stop_sequence;
		\end{lstlisting}
	\end{minipage}
	\caption{K\^od pogleda \textit{triptime}}
	\label{slk:triptime}
\end{figure}

Moguće je direktno pristupati virtualnim tablicama, ali za potrebe aplikacije potrebno je još filtrirati ih prema određenom parametru, zato u bazi podataka imamo i funkcije definirane u jeziku\textit{ plpgsql}.
One su: 
\begin{itemize}
	\item \textit{\textbf{find\_route\_between\_stops (}stop\_a, stop\_b\textbf{)}} - prima dva parametra (dvije stanice) i vraća listu ruti koje prolaze kroz te dvije stanice.
	\item \textit{\textbf{get\_trip\_info (}trip\_id\textbf{)}} - prima parametar \textit{trip\_id} i vraća listu svih stanica puta, nazive i lokacije stanica, vremena dolaska i odlaska te vrijeme potrebno do sljedeće stanice, koristi se za prikaz informacija klikom na marker.
	\item \textit{\textbf{get\_trip\_details (}trip\_id\textbf{)}} - prima parametar \textit{trip\_id} i vraća puni naziv rute i smjer.
\end{itemize}

\begin{figure}[h]
	\centering
	\begin{minipage}{0.8\linewidth}
		\begin{lstlisting}[language=SQL]
SELECT jsonb_agg(
	jsonb_build_object(
	'stop_id', stop_id,
	'stop_sequence', stop_sequence,
	'stop_name', stop_name,
	'stop_headsign', stop_headsign,
	'stop_lat', stop_lat,
	'stop_lon', stop_lon,
	'arrival_time', arrival_time,
	'departure_time', departure_time,
	'time_till_next_stop', time_till_next_stop::interval
	)
) INTO trip_info
FROM (
	SELECT *,
	COALESCE(
		LEAD(arrival_time) OVER (ORDER BY stop_sequence) - arrival_time, INTERVAL '0 seconds'
	) AS time_till_next_stop

		\end{lstlisting}
	\end{minipage}
	\caption{Isječak k\^oda funkcije \textit{get\_trip\_info(trip\_id)}}
	\label{slk:sqlfunkcija}
\end{figure}

Navedene funkcije koristi poslužiteljska strana zajedno s parametrom.
\begin{figure}[h]
	\centering
	\begin{minipage}{0.9\linewidth}
		\begin{lstlisting}[language=SQL]
		SELECT find_routes_between_stops('Prisavlje', 'Miramarska') AS routes
		SELECT get_trip_info('0_21_26820_268_10001') AS trip_info
		\end{lstlisting}
	\end{minipage}
	\caption{Primjer korištenja funkcija}
	\label{slk:sqlprimjer}
\end{figure}


\section{Pomoćna skripta za dohvaćanje Zet-ovih podataka}
\label{sec:skripta}
Za potrebe aplikacije izrađena je jedna pomoćna skripta koju pokreće poslužiteljski dio aplikacije prilikom pokretanja. Ona je zadužena za dohvaćanje najnovijih datoteka sa ZET-ove stranice te je napisana u Pythonu koristeći module \textit{psycopg2, requests i zipfile}.\\\\
Pomoću metode \textit{requests.get('https://www.zet.hr/gtfs-scheduled/latest')} sa stranice ZET-a dohvaća najnoviju \textit{GTFS static} arhivu te potom metodom \textit{extractAll} iz modula \textit{zipfile} raspakirava arhivu i sprema datoteke. Komunikacija s bazom podataka \textit{PostgreSQL} je ostvarena metodama iz modula \textit{psycopg2}. Prvo je potrebno priložiti vjerodajnice odnosno \textit{username i password} te \textit{host, port i database} (određuju lokaciju baze podataka) za uspješno spajanje.\\\\
Na slici \ref{slk:pytonskripta} prikazan je isječak k\^oda pomoću kojeg je ostvareno učitavanje podataka iz datoteka u odgovarajuće tablice. Python funkcije rade u tri koraka, prvo definiramo SQL naredbu zatim šaljemo tu naredbu bazi podataka i na kraju gledamo odgovor je li ispravno izvršena. Funkcija \textit{\textbf{load\_csv\_to\_postgres}(connection, cursor, 'routes', 'routes.txt')} učitava podatke iz \textit{routes.txt}  i sprema ih u bazu podataka kao tablica s imenom \textit{routes}, a funkcija \textit{\textbf{create\_route\_find\_table}(connection, cursor)} kreira tablicu \textit{route\_find} koju koristi funkcija \textit{\textbf{find\_route\_between\_stops}(stop\_a, stop\_b)} koja pronalazi rute koje prolaze kroz neke dvije stanice.

\begin{figure}[H]
	\centering
	\begin{minipage}{0.9\linewidth}
		\begin{lstlisting}[language=Python]
		def load_csv_to_postgres(connection, cursor, table_name, csv_file_path):
			try:
				drop_table_sql = f'TRUNCATE {table_name} CASCADE;'
				execute_sql(connection, cursor, drop_table_sql)

				copy_sql = f""" COPY {table_name} FROM '{csv_file_path}' WITH
						CSV HEADER DELIMITER as ','"""
				execute_sql(connection, cursor, copy_sql)
				print(f"Loaded {table_name}")
			except Exception as e:
				print(f"Error loading {table_name}: {e}")

		def create_route_find_table(connection, cursor):
			create_table_sql = """ CREATE TABLE route_find AS
					SELECT trips.route_id, trips.trip_id, stop_times.stop_sequence,
						   stop_id, stops.stop_name
					FROM stop_times
					JOIN trips USING (trip_id)
					JOIN stops USING (stop_id); """
			# SQL naredba za brisanje stare tablice i stvaranje indeksa
			try:
				# execute naredbe
				execute_sql(connection, cursor, create_table_sql)

				print("route_find table created successfully.")
			except Exception as e:
				print(f"Error creating route_find table: {e}")
		\end{lstlisting}
	\end{minipage}
	\caption{Isječak k\^oda skripte}
	\label{slk:pytonskripta}
\end{figure}
\newpage
\section{Poslužiteljski dio aplikacije - node.js}
Poslužiteljski dio aplikacije napisan je u jeziku JavaScript, koristeći okvir \textit{node.js}. JavaScript je programski jezik primarno namijenjen korištenju unutar web preglednika, ali uz pomoć \textit{node.js} moguće je izvršavanje JavaScript k\^oda na poslužiteljskoj strani, izvan web preglednika, čime se omogućava korištenje jednog jezika za pisanje cjelokupne aplikacije. Najvažniji dodatak za \textit{node.js} je \textit{Express.js}, minimalistički i fleksibilan web okvir koji olakšava rad s HTTP zahtjevima i izgradnju web poslužitelja. On omogućava jednostavno definiranje ruta, rukovanje HTTP zahtjevima i upravljanje \textit{middleware} funkcijama.\\
Na slici \ref{slk:express} je primjer vrlo jednostavnog poslužitelja koji kada dobije GET zahtjev na ruti \textit{'/api/stops'} provodi upit nad bazom podataka i ovisno o odgovoru vraća uspješan (200 OK s podacima o stanicama) ili neuspješan odgovor (status 500 s porukom pogreške). Sve ostale dostupne rute aplikacije su napravljene prema istoj shemi.

\begin{figure}[H]
	\centering
	\begin{minipage}{0.9\linewidth}
		\begin{lstlisting}[language=JavaScript]
			const express = require('express');
			const app = express();
			
			app.get('/api/stops', async (req, res) => {
				try {
					const result = await pool.query(`SELECT DISTINCT stop_name FROM stops`);
					res.json(result.rows);
				} catch (error) {
					console.error('[Error] Executing query', error);
					res.status(500).json({ error: 'An error occurred' });
				}
			});
			
			app.listen(8080, () => {
				console.log('[Info] Server started on port 8080');
			});
		\end{lstlisting}
	\end{minipage}
	\caption{Primjer k\^oda jednostavnog poslužitelja}
	\label{slk:express}
\end{figure}
\newpage
Kao i kod pomoćne skripte i na poslužitelju je potrebno ostvariti vezu s bazom podataka, to je omogućeno s modulom \textit{pg}. Taj modul omogućava upravljanje vezama prema bazi i izvršavanje upita.
Pomoću objekta \textit{pool} definiramo potrebne informacije za spajanje poput \textit{username, host, database, password i port} te onda pozivom metode \textit{query} nad objektom \textit{pool} možemo izvesti proizvoljne SQL-naredbe.

\begin{figure}[H]
	\centering
	\begin{minipage}{0.9\linewidth}
		\begin{lstlisting}[language=JavaScript]
			const { Pool } = require('pg');
			require('dotenv').config();

			const pool = new Pool({
				user: process.env.DB_USER,
				host: process.env.DB_HOST,
				database: process.env.DB_NAME,
				password: process.env.DB_PASSWORD,
				port: process.env.DB_PORT,
			});
			
			pool.query('SELECT * FROM stops');
		\end{lstlisting}
	\end{minipage}
	\caption{Primjer spajanje s bazom podataka}
	\label{slk:baza}
\end{figure}

Poslužitelj osim poslova upravljanja zahtjevima i komunikacijom s bazom podataka radi najbitniji dio aplikacije, a to je dohvaćanje \textit{GTFS realtime} podataka sa ZET-ove stranice. Taj dio je ostvaren kao funkcija koja se periodički poziva (svakih 15 sekundi) kako bi dohvatila najnoviju \textit{.protobuf} datoteku s realtime podacima, obradila ju i spremila u memoriju za brži pristup. Na slici \ref{slk:dohvatGTFS} je čitav kod koji je zadužen za dohvaćanje podataka i njegovu deserijalizaciju iz binarnog .protobuf formata u JSON oblik.\\

Ti podaci se tada spremaju u jedan objekt \textit{data} koji je napravljen tako da sadrži brojeve ruta kao ključeve i svaki tramvaj svrstava u točnu grupu. Također postoji funkcija koja upravlja dodavanjem u taj objekt i funkcija koja periodički briše tramvaje. To je napravljeno jer ZET-ovi podaci nisu konzistentni odnosno neki tramvaji nekada samo nestanu iz podataka makar je u \textit{FeedHeader}-u \textit{.protobuf} poruke navedeno da se svaki put šalju svi podaci (cijeli \textit{dataset}).

\begin{figure}[H]
	\centering
	\begin{minipage}{0.9\linewidth}
		\begin{lstlisting}[language=JavaScript]
const GtfsRealtimeBindings = require('gtfs-realtime-bindings');
async function fetchData() {
	try {
		const response = await fetch(feedUrl);
		if (!response.ok) {
			const error = new Error(
			`${response.url}: ${response.status} ${response.statusText}`
			);
			error.response = response;
			throw error;
		}
		const buffer = await response.arrayBuffer();
		const feed = GtfsRealtimeBindings.transit_realtime.FeedMessage.decode(
		new Uint8Array(buffer)
		);
		
		feed.entity.forEach((entity) => {
			if (entity) parseEntity(entity);
		});
	} catch (error) {
		console.log(error);
	}
}
		\end{lstlisting}
	\end{minipage}
	\caption{K\^od za dohvaćanje \textit{GTFS realtime} podataka i deserijalizaciju}
	\label{slk:dohvatGTFS}
\end{figure}
\newpage

\section{Klijentski dio aplikacije - React.js}
\label{sec:frontend}

Klijentski dio također je izrađen u jeziku JavaScript pomoću \textit{React.js}, popularnog JavaScript okvira koji se primarno koristi za izgradnju jednostraničnih aplikacija (SPA). \textit{React.js} omogućava razvoj složenih i dinamičnih web aplikacija pomoću komponentnog pristupa, gdje se korisničko sučelje sastoji od manjih, ponovno upotrebljivih komponenti.\\
Komponente su osnovni gradivni elementi u okviru \textit{React.js}, a svaka komponenta predstavlja dio korisničkog sučelja. Komponente mogu biti jednostavne, poput gumba, ili složene, poput cijele stranice. Komponente su definirane pomoću JavaScript funkcija ili klasa, te vraćaju JSX (JavaScript XML), koji izgleda slično HTML-u.\\

Aplikacija se sastoji od nekoliko komponenti \textit{App.js, Map.js, MapRouteLine.js, Marker.js, Sidebar.js i TripTimeline.js} od kojih je glavna roditeljska komponenta koja sadrži sve ostale \textit{App.js} (slika \ref{slk:strukturaklijent}).

\begin{figure}[H]
	\dirtree{%
		.1 \textbf{client}.
		.2 \textbf{node\_modules}.
		.2 \textbf{public} \hspace{2em} -Kazalo s slikama ikona.
		.2 \textbf{src}.
		.3 \textbf{App.css} \hspace{2em}-Glavni stylesheet za aplikaciju.
		.3 \textbf{App.js} \hspace{2em} -Glavna React komponenta.
		.3 \textbf{Map.js} \hspace{2em} -Komponenta za kartu.
		.3 \textbf{MapRouteLine.js} \hspace{2em} -Komponenta za linije ruta na karti.
		.3 \textbf{Markers.js} \hspace{2em} -Komponenta za markere na karti.
		.3 \textbf{Sidebar.js} \hspace{2em} -Komponenta za izbornik.
		.3 \textbf{TripTimeline.js} \hspace{2em} -Komponenta za vremensku liniju putovanja.
		.2 \textbf{package-lock.json}.
		.2 \textbf{package.json}.
		.2 \textbf{README.md}.
	}
	\caption{Osnovna struktura klijentskog kôda}
	\label{slk:strukturaklijent}
\end{figure}

\subsubsection{Komponenta App.js}
Kao što je rečeno \textit{App.js} je glavna komponenta koja se nalazi iznad svih ostalih, njezin glavni posao je da objedinjuje komponente i povremeno dohvaća (osvježava) podatke od poslužiteljskog dijela aplikacije uz pomoć \textit{fetch} naredbe. Pseudok\^od na slici \ref{slk:dohvatpodataka} prikazuje način dohvaćanja podataka.\\ K\^od iskorištava \textit{useEffect() i useState()} hook-ove. Hook-ovi su način na koji React omogućuje dinamičko upravljanje stanjem i ponašanjem funkcionalnih komponenti i njihovim životnim ciklusom. Hook \textit{useEffect()} omogućuje komponentama da definiraju popratne efekte koji se trebaju izvesti nakon što se promjeni neko stanje definirano \textit{useState()} hook-om. U k\^odu aplikacija svakih 10 sekundi ili kada se promjeni stanje s odabranim rutama dohvaća nove podatke. Za svaku pojedinu navedenu rutu izvodi se jedan GET zahtjev, to se izvodi asinkrono pomoću obećanja (\textit{Promises}) te se na kraju kada su svi zahtjevi obrađeni grupira zajedno i šalje \textit{"child"} komponenti \textit{Map} koja prikazuje tramvaje kao markere na karti.


\begin{figure}[H]
	\centering
	\begin{minipage}{0.9\linewidth}
		\begin{lstlisting}[language=JavaScript]			
			useEffect(() => {
				if (route && route.length > 0) {
					const intervalId = setInterval(async () => {
							const allRouteData = await Promise.all(
								route.map(fetchDataForRoute)
							);
							const combinedData = allRouteData.flat();
							setRouteData(combinedData);
					}, 10000);
					
					return () => clearInterval(intervalId);
				} else {setRouteData(null);}
			}, [route, setRouteData]);
			
			const fetchDataForRoute = async (routeValue) => {
				const response = await fetch(
						`/api/route/${routeValue}`
				);
				...
			};
		\end{lstlisting}
	\end{minipage}
	\caption{Pseudok\^od dohvaćanja podataka od poslužitelja}
	\label{slk:dohvatpodataka}
\end{figure}

\subsubsection{Komponenta Map.js}
Komponenta \textit{Map.js}, kojoj je glavna uloga da prikazuje kartu, ostvarena je pomoću \textit{\textit{Leaflet.js}}-a.
\textit{\textit{Leaflet.js}} je popularna biblioteka otvorenog kôda pisana u jeziku JavaScript za izradu interaktivnih karti. Omogućava jednostavno dodavanje dinamičkih karti, postavljanje markera, crtanje linija i poligona, interakciju s kartama, dodavanje pop-up prozora i još mnogo toga \cite{leaflet}.
Postoji \textit{React.js} verzija \textit{\textit{Leaflet.js}} koja objekte tretira kao komponente i s kojom je još jednostavnije raditi u Reactu.\\
Karta koju \textit{\textit{Leaflet.js}} koristi je dohvaćena iz OpenStreetMap projekta, koji je otvorena i slobodna baza geografskih podataka koju održavaju volonteri.\\

Na slici \ref{slk:mapjs} nalazi se isječak k\^oda koji opisuje što sadrži \textit{Map.js} te možemo primijetiti standardne komponente iz biblioteke \textit{Leaflet.js} koje definiraju izgled karte: \textit{MapContainer}, \textit{ZoomControl} i \textit{TileLayer}.

Pored njih tu je i komponenta \textit{Markers} koja se koristi za postavljanje markera na kartu i komponenta \textit{MapRouteLine} koja se koristi za crtanje linija ruta na karti su definirane za potrebe aplikacije.

\begin{figure}[H]
	\centering
	\begin{minipage}{0.9\linewidth}
		\begin{lstlisting}[language=JSX]
	return (
		<MapContainer
		center={[45.808680463038435, 15.977835680373971]}
		zoom={13}
		zoomControl={false}
		whenReady={(event) => setMapContext(event.target)}
		>
		<ZoomControl position="bottomright" />
		<TileLayer
		attribution='&copy; <a href="https://www.openstreetmap.org/copyright">OpenStreetMap</a> contributors'
		url="https://tile.openstreetmap.org/{z}/{x}/{y}.png"
		/>
		<Markers
		routeData={routeData}
		tripInfo={tripInfo}
		....
		/>
		<MapRouteLine coordinates={coordinates} />
		<MapClickHandler />
		</MapContainer>
	);
		\end{lstlisting}
	\end{minipage}
	\caption{Isječak k\^oda Map.js}
	\label{slk:mapjs}
\end{figure}

\subsubsection{Komponenta Markers.js}
\textit{Markers.js} komponenta je zadužena za stvaranje marker na karti koji će prikazivati pozicije tramvaja i njihove dotične animacije. Sami markeri na karti su ostvareni uz pomoć \textit{react-\textit{Leaflet.js}} komponenti \textit{Marker i Popup}. Animacije markera koje predstavljaju predikcije položaja tramvaja su bazirane na vremenima iz ZET-ovih rasporeda koji su dio static datoteka (poglavlje \ref{sec:zet-gtfs}). Osim prikaza obrađuje i klikove na pojedine markere koji tada aktiviraju popratne akcije prikaza dodatnih informacija o putu pomoću \textit{Popup i TripTimeline} komponenti.

\subsubsection{Komponenta MapRouteLine.js}
\textit{MapRouteLine.js} ostvaruje iscrtavanje putanje tramvaja na karti dohvaćanjem pozicija stanica tramvaja i spajanjem tih koordinata u jednu \textit{Polyline} liniju. Idealno rješenje ovog problema bi bili podaci iz datoteke \textit{shapes.txt} koji detaljno propisuju putanju rute, ali ta datoteka nedostaje u ZET-ovim static podacima.
Još jedan način rješavanja ovog problema je korištenjem \textit{routing} komponente kojoj predajemo koordinate stanica te onda ona provodi pronalaženje puta na karti između svake dvije točke i time stvara glatku putanju na karti. Problem kod tog načina je da se za pronalaženje puta koriste putevi kojim auti mogu voziti, a ne tračnice kojim tramvaji voze što također dovodi do nepreciznosti tjst. neistinitosti nacrtane putanje naspram stvarne, tako da je taj način odbačen.

\subsubsection{Komponenta Sidebar.js}
\textit{Sidebar.js} je glavna i jedina komponenta koja ostvaruje komunikaciju s korisnikom. Korisniku su dostupne padajuće liste ostvarene komponentom \textit{Select} kojima bira rute ili stanice kod dijela za pretraživanje ruta. Također sadrži nekoliko tipki koje služe za upravljanjem nad odabranim rutama. Kada se odabere ruta ili pretraže rute za stanicu komponenta mijenja stanje aplikacije što izaziva dohvaćanje novih podataka.

\subsubsection{Komponenta TripTimeline.js}
Klikom na marker se izaziva \textit{useEffect()} hook u komponenti \textit{TripTimeline.js} koje aktivira njen prikaz. Ta komponenta je zadužena za prikazivanje rasporeda puta tramvaja, koje je ostvareno kao vremenska crta pomoću \textit{VerticalTimeline} komponente. Raspored puta sadrži u redoslijedu stanice na koje će tramvaj doći, njihovo vrijeme očekivanog dolaska i vremena potrebnog do sljedeće stanice.

\newpage
\chapter{Izgled aplikacije}
Prilikom prvog pristupa stranici korisniku se prikazuje prazna karta bez markera. U gornjem lijevom kutu nalazi se glavni izbornik koji se sastoji od padajućih izbornika i polja koji sadrži listu odabranih ruta (to polje je vidljivo samo kada je barem jedna ruta odabrana). Svi padajući izbornici ujedno podržavaju i pretraživanje.\\
Desno od glavnog izbornika nalazi se analogni sat za vrijeme koji se može promijeniti u digitalni sat klikom na njega. Ispod sata se nalazi legenda markera uz pomoć koje se može vidjeti što koji marker predstavlja kada je miš iznad odgovarajuće ikone ("\textit{on hover}"). Također, ispod legende nalaze se i tipke za uvećavanje/smanjivanje karte, ali to je moguće i lakše obaviti kotačićem na mišu.

\begin{figure}[H]
	\centering
	\includegraphics[width=\linewidth]{Figures/default.png} 
	\caption{Početni zaslon aplikacije}
	\label{slk:default}
\end{figure}
\newpage
Korištenjem padajućeg izbornika označenog tekstom \textit{"Odaberi rutu:"} možemo ili klikom odabrati rutu koju želimo pratiti ili jednostavno upišemo broj tramvajske linije i pritisnemo tipku \textit{enter} (slika \ref{slk:glavni}). Tada će odabrana ruta biti dodana u kutiju odabranih ruta (slika \ref{slk:odabrane}) i na karti će se prikazati markeri tramvaja.

Aplikacija sadrži funkcionalnost pronalaska tramvajskih linija odnosno funkcionalnost koja pomaže pronaći tramvajske linije koje prolaze kroz početnu i krajnju tramvajsku stanicu. Time olakšava uporabu korisnicima koji nisu upoznati s voznim redom tramvajskih linija. Pomoću dva padajuća izbornika (slika \ref{slk:pretrazivanje}) potrebno je odabrati željenu početnu i krajnju stanicu te klikom na tipku \textit{"Pretraži"} obavlja se pretraga. Ako je pretraga bila uspješna stanice će se automatski dodati u polje odabranih stanica, a ako nije bila uspješna tada će se prikazati poruka \textit{"Nije pronađena nijedna linija!"} koja će nakon 10 sekundi nestati.

Polje sa odabranim tramvajskim linijama za svaku liniju ima jedan gumb za brisanje te linije \textit{"Obriši"}. Ako je odabrano više od 3 tramvajske linije tada će se prikazati i dodatan gumb \textit{"Obriši sve"} s kojim lakše i brže brišemo sve odabrane linije.\\

\begin{figure}[H]
	\centering
	\begin{minipage}[c]{0.3\textwidth}
		\centering
		\includegraphics[width=\textwidth]{Figures/glavni.png}
		\caption{Glavni izbornik}
		\label{slk:glavni}
	\end{minipage}
	\hfill
	\begin{minipage}[c]{0.3\textwidth}
		\centering
		\includegraphics[width=\textwidth]{Figures/pretrazivanje.png}
		\caption{Izbornik za pronalaženje ruta}
		\label{slk:pretrazivanje}
	\end{minipage}
	\hfill
	\begin{minipage}[c]{0.3\textwidth}
		\centering
		\includegraphics[width=\textwidth]{Figures/odabrani.png}
		\caption{Odabrane rute}
		\label{slk:odabrane}
	\end{minipage}
\end{figure}
\newpage
Kada smo odabrali neku od linija na zaslonu će se prikazati markeri tramvaja (slika \ref{slk:zaslonmark}). 
Značenja boja markera tramvaja su:
\begin{figure}[H]
	\centering
	\begin{minipage}[b]{0.68\textwidth}
		\begin{itemize}
			\item Crni marker - Zadnja potvrđena lokacija tramvaja
			\item Zeleni marker - Predikcija lokacije tramvaja
			\item Plavi marker - Odabrani tramvaj
			\item Ljubičasti marker - Predikcija odabranog tramvaja
		\end{itemize}
	\end{minipage}
	\hfill
	\begin{minipage}[c]{0.24\textwidth}
		\centering
		\includegraphics[width=\textwidth]{Figures/legenda.png}
		\caption{Legenda}
		\label{slk:legenda}
	\end{minipage}
\end{figure}

\begin{figure}[H]
	\centering
	\includegraphics[width=\linewidth]{Figures/odabrantram.png} 
	\caption{Zaslon s markerima odabranih linija}
	\label{slk:zaslonmark}
\end{figure}
Klikom na bilo koji od crnih markera, odabrani marker će poplaviti i iznad njega će se otvoriti mali oblačić s informacijama o trenutnoj stanici te će se odgovarajući marker predikcije lokacije tramvaja promijeniti u ljubičastu boju za lakše uočavanje. Ako želimo odznačiti marker, možemo pritisnuti bilo gdje drugdje na karti ili na neki drugi marker.\\
Također, na karti će se iscrtati cjelokupni put tramvajske linije i otvorit će se novi prozor s puno više informacija o voznom redu.
\begin{figure}[H]
	\centering
	\includegraphics[width=\linewidth]{Figures/tram1.png} 
	\caption{Zaslon s odabranim tramvajem}
	\label{slk:tramodabran}
\end{figure}

\begin{figure}[H]
	\centering
	\includegraphics[width=\linewidth]{Figures/primjer1.png} 
	\caption{Zaslon s iscrtanim putem}
	\label{slk:put}
\end{figure}
\newpage
Prozor s dodatnim informacijama o putu (slika \ref{slk:inf}) je vidljiv samo kada je jedan od tramvaja odabran. On sadrži informacije o broju i punom nazivu tramvajske linije koji se sastoji od početne i krajnje stanice te smjera kretanja tramvaja i sadrži vremensku liniju voznog rasporeda tramvaja.\\

Raspored tramvaja sadrži sve stanice na tramvajskoj liniji i informacije o očekivanom vremenu dolaska na pojedinu stanicu te o potrebnim vremenima do sljedeće stanice. Ta vremena su temeljena na povijesnim podacima koji se nalaze unutar ZET-ovih static podataka. Kroz raspored se može proći uporabom kotačića na mišu i time vidjeti sve prijašnje i naredne stanice.

\begin{figure}[H]
	\centering
	\includegraphics[width=0.4\linewidth]{Figures/informacije.png} 
	\caption{Prozor s voznim redom tramvaja}
	\label{slk:inf}
\end{figure}

Pomoću stila odnosno uporabom Cascading Style Sheets (CSS), napravljena je podrška za responzivni dizajn aplikacije. To omogućuje da se aplikacija prilagodi različitim veličinama zaslona mobilnih uređaja, tableta i stolnih računala. Na slici \ref{fig:mobiteli} je prikazan izgled sučelja na rezoluciji koja se može pronaći na mobilnim uređajima.
\begin{figure}[H]
	\centering
	\begin{minipage}[c]{0.45\linewidth}
		\centering
		\includegraphics[width=\textwidth]{Figures/mobitel.png}
		\caption*{(a)}
		\label{slk:mobitel1}
	\end{minipage}
	\hfill
	\begin{minipage}[c]{0.45\linewidth}
		\centering
		\includegraphics[width=\textwidth]{Figures/mobitel2.png}
		\caption*{(b)}
		\label{slk:mobitel2}
	\end{minipage}
	\caption{Izgled aplikacije na manjim zaslonima}
	\label{fig:mobiteli}
\end{figure}

%--- ZAKLJUČAK / CONCLUSION ----------------------------------------------------
\chapter{Zaključak}
\label{pog:zakljucak}
Aplikacija opisana u ovom završnom radu površinski opisuje jedan način izrade aplikacije za praćenje tramvaja te se može smatrati prvim prototipom. Mjesto za unaprijeđenje postoji, neki ključni dijelovi koji bi doveli do unaprijeđenja su ZET-ova potpuna implementacija GTFS podataka i izgradnja novog prototipa s novim znanjima stečenim tijekom istraživanja i izrade ovog rada. Ti podaci bi omogućili još više naprednijih mogućnosti poput planiranja putovanja i upravljanja presjedanjima te bi pružili detaljnije informacije o svakoj tramvajskoj liniji i stanici. Glavna zadaća ove aplikacije je bila praćenje tramvaja za grad Zagreb, ali to ne bi trebalo sprječavati razvoj aplikacije i u druge svrhe poput praćenja različitih javnih prijevoza u drugim gradovima tjst. trebala bi se moći primijeniti na svaki izvor podataka koji odgovora GTFS formatu.\\
Ova aplikacija ima mogućnost uvelike poboljšati život gradskog stanovništva pružajući im detaljne informacije o javnom prijevozu, što olakšava planiranje putovanja. Također, turistima bi mogla olakšati kretanje kroz njima nov i nepoznat grad uporabom javnog prijevoza, što ima potencijal razviti grad i ekonomiju te će sigurno potaknuti daljnje unaprijeđenje javnog prijevoza.

%--- LITERATURA / REFERENCES ---------------------------------------------------

% Literatura se automatski generira iz zadane .bib datoteke / References are automatically generated from the supplied .bib file
% Upiši ime BibTeX datoteke bez .bib nastavka / Enter the name of the BibTeX file without .bib extension

\nocite{*}

\bibliography{literatura}

%--- SAŽETAK / ABSTRACT --------------------------------------------------------

% Sažetak na hrvatskom
\begin{sazetak}
	Ovaj završni rad prikazuje izradu aplikacije za praćenje tramvaja u gradu Zagrebu za vizualizaciju i predviđanje položaja tramvaja u stvarnom vremenu. Aplikacija se temelji na GTFS-podacima koji omogućuje dijeljenje informacija javnog prijevoza i popratnim tehnologijama poput \textit{Protocol buffers}. Detaljno se analiziraju podaci Zagrebačkog električnog tramvaja koji su ključni za izradu aplikacije te na temelju kojih definiramo bazu podataka te poslužiteljski i klijentski dio aplikacije koji čine cjelokupnu sliku arhitekture aplikacije. Detaljno prolazimo kroz ključne dijelove programskog k\^oda koji omogućuju temeljene funkcionalnost i rad aplikacije.
	Na kraju opisujemo i prolazimo kroz ostvareno rješenje i izvodimo zaključak koji opisuje daljnja unaprijeđenja aplikacije.
\end{sazetak}

\begin{kljucnerijeci}
  Završni rad; Web aplikacija; Zagrebački električni tramvaj; \textit{GTFS static}; \textit{GTFS realtime}; Protobuf; React; \textit{node.js}; Praćenje tramvaja; \textit{Leaflet.js}
\end{kljucnerijeci}


% Abstract in English
\begin{abstract}
	This final thesis follows the creation of an application for tracking trams in the city of Zagreb and explores the necessary parts to achieve this goal. Through work, we get to know the GTFS format which enables the sharing of public transport information and accompanying technologies such as \textit{Protocol buffers}. We analyze in detail the data provided by \textit{Zagrebački Električni Tramvaj} that are key to creating the application and on the basis of which we define the database and the server and client part of the application that make up the overall image of the application architecture. We go in detail through the key parts of the code that make base functionalities and the whole operation of the application possible. Finally, we describe and go through the solution and draw a conclusion that describes further improvements of the application.
\end{abstract}

\begin{keywords}
  Final thesis; Web application; Zagreb electric tram; \textit{GTFS static}; \textit{GTFS realtime}; Protobuf; React; \textit{node.js}; Tram monitoring; \textit{Leaflet.js}
\end{keywords}


%--- PRIVITCI / APPENDIX -------------------------------------------------------

% Sva poglavlja koja slijede će biti označena slovom i riječi privitak / All following chapters will be denoted with an appendix and a letter
\backmatter

\chapter{K\^od}
Cjelokupni kod aplikacije i dokumentacija može se pronaći online preko Github-a, gdje je dostupan za pregled i preuzimanje. Izravna poveznica na repozitorij je\\ \textit{https://github.com/Luka147m/Zavrsni-rad}.


\end{document}
